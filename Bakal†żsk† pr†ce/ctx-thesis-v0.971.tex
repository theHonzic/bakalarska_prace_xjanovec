% macros=mkvi

%====================================================================%
% CTX-THESIS ver. 0.97 -- Thesis in ConTeXt                          %     
%--------------------------------------------------------------------%
% Sazební styl pro ConTeXt -- (c) TH, 2013--2022                     %
%--------------------------------------------------------------------%
% t. č. ve vývoji                                                    %
%====================================================================%
% 0.971 20220506 ~
% 0.97 20220408 \swapfrontmatterpages; pt=projekt; degree; \obrazekB; English macronames: \figure*; missing lang. redir. to en ; lot
% 0.96 20211127 \až, \aŽ; TODO+QUES: \par; 
% 0.95 20211102 \ORIGmainlanguage; endash; v/v.; hyphexcps: us/uk; userdata-->documentdata; \*obrazek* v list v autocaptionsc etc.
%				interaction: def. values; THESISlistof: revision
% 0.94 20210512 vs; pppkap[x]: vspace 2:1; ppppkap[x]: corr.; THESISlist: charts; mieste; hyph.
% 0.93 20210419 src: get v type; THESISlists
% 0.92 20210417 THESISlists+headtext: src; hyph. přespal
% 0.91 20210416 new itemize system; cz-->cs; floatsrccode, corr: pp, desc, THESISlist
% 0.90 20210403 hyphenation; trim-->strip; decl:sk; setup ppkap+x, ppppkap+x; pdf2ab; \scr ; corr.: THESISlists, {\\,}; fig: plurals

%----------- Auxillary macros ---------------------------------------%

\def\titnl{\crlf}
\def\tocnl{\space}
\def\extrapar{\godown[-14dd]~\par}  %% kvůli chybě u bottom
\def\dirtypar{\crlf\null\hskip1.25em}
\let\uv=\quotation
\def\citat#1{{\it\uv{#1}}}
\def\,{\kern0.175em}
%\def~{\penalty10000\ }
\def\edots{\sixperemspace\dots}
\def\sdots{\edots\sixperemspace}
\def\sDots{\edots\ }
\def\pojem#1{{\bf#1}}
\def\akol{\unskip\ a~kol.\doifnextcharelse{(}{\ }{}}%\ifx\undefined\BIBSTYversion~\fi}
\def\etal{\unskip\ et~al.\doifnextcharelse{(}{\ }{}}%ifx\undefined\BIBSTYversion~\fi}
\ifx\undefined\mycshyphen\def\={\discretionary{-}{-}{-}}\fi
\def\bye{\stopbodymatter\stopthesis\endinput}
\definecolor[lightred][r=1,g=0.7,b=0.7]
\defineframedtext[framedtextTODO][frame=on,offset=1cc,backgroundcolor=lightred]
\defineframedtext[framedtextQUES][frame=on,offset=1cc,backgroundcolor=yellow]
\def\TODO#1\par{\par\framedtextTODO{#1}\par}
\def\QUES#1\par{\par\framedtextQUES{#1}\par}

%----------- CSV definitions ------- --------------------------------%

\ifx\undefined\defineseparatedlist\usemodule[database]\fi
\defineseparatedlist[CSV][before=\bTABLE,after=\eTABLE,first=\bTR,last=\eTR,left=\bTD,right=\eTD,separator=;]
\defineseparatedlist[LTX][before=\bTABLE,after=\eTABLE,first=\bTR,last=\eTR,left=\bTD,right=\eTD,separator=&]

%----------- Auxillary Lua functions --------------------------------%

\startluacode
table.getn=function (t) return #t end
string.mysplit = function (s,sep) if s and s:strip()~="" then t=s:split(sep) else t={} end return t end
tex2txt = function(s) s=s:gsub("~"," ") s=s:gsub("\\","") return s:strip() end
\stopluacode

%----------- Languages : hyphenation ----------------------------------%

\definebreakpoints[cscompound]
\definebreakpoint[cscompound][-][nleft=3,nright=3,type=4]

\startluacode
fonts.protrusions.classes.cscompound = { 
    vector = 'cscompound', factor = 1,
}
fonts.protrusions.vectors.cscompound = table.merged (
    fonts.protrusions.vectors.quality,
    {
        [0x002D] = { 0.4, 0.7 }, -- hyphen
    }
)
\stopluacode

%----------- Languages : setups, dashes ------------------------------%

\def\THESISdashes{
  \let\az=\idash
  \let\aZ=\Idash 
  \let\až=\az
  \let\aŽ=\aZ
  \def\ip{\pdash}
  \let\vs=\vdash
  \let\vS=\Vdash
  \let\VS=\VdAsh
}

\def\THESISdefaultitems{
  \setupitemize[1][symbol=1]
  \setupitemize[2][symbol=2]
  \setupitemize[3][symbol=3]
  \setupitemize[4][symbol=4]
  \setupitemize[5][symbol=5]
}

\def\THESISczechitems{
  \setupitemize[1][symbol=2]
  \setupitemize[2][symbol=5]
  \setupitemize[3][symbol=4]
  \setupitemize[4][symbol=1]
  \setupitemize[5][symbol=3]
}

\startsetups [ukenglish]
  \resetbreakpoints
  \def\idash{\discretionary{\char32to}{}{\endash}}
  \def\Idash{\discretionary{}{to\char32}{\endash}}
  \def\pdash{ \endash\ }
  \def\vdash{\discretionary{\char32versus}{}{\endash}}
  \def\Vdash{\discretionary{}{versus\char32}{\endash}}
  \def\vdAsh{\discretionary{\char32v}{}{\endash}}
  \def\VdAsh{\discretionary{}{v\char32}{\endash}}
  \def\akol{\etal}
  \THESISdashes
  \THESISdefaultitems
\stopsetups

\startsetups [usenglish]
  \setuplanguage[en][setups=english]
  \def\pdash{\emdash}
%  \let\ip=\pdash
  \THESISdashes
\stopsetups

\startsetups [czech]
  \setbreakpoints [cscompound]
  \def\idash{\discretionary{\char32až}{}{\endash}}
  \def\Idash{\discretionary{}{až\char32}{\endash}}
  \def\pdash{~\endash\ }
  \def\vdash{\discretionary{\char32versus}{}{\endash}}
  \def\Vdash{\discretionary{}{versus\char32}{\endash}}
  \def\vdAsh{\discretionary{}{vs.\char32}{\endash}}
  \let\VdAsh=\vdAsh
  \THESISdashes
  \THESISczechitems
\stopsetups


\setuplanguage[cs][setups=czech]
\setuplanguage[sk][setups=czech]
\setuplanguage[uk][setups=english]
\setuplanguage[en][setups=usenglish]

\let\ORIGmainlanguage=\mainlanguage
\def\mainlanguage[#1]{%
%!!! #1 !!!
   \doifinsetelse{#1}{cs,cz,czech,sk,slovak}%
 	  {\setuplanguage[cs][setups=czech]
	  }{%
   \doifinsetelse{#1}{en-gb,gb,uk,ukenglish}%
      {\setuplanguage[en-gb][setups=ukenglish]
	  }{%
   \doifinsetelse{#1}{en,us,usenglish}%
      {\setuplanguage[us][setups=usenglish]
      }{\setuplanguage[#1][setups=english]}
     }} \ORIGmainlanguage[#1]
   }

%----------- Languages : hyphenation exceptions ---------------------%

\definefontfeature [default] [default] [protrusion=cscompound]

\mainlanguage[cs]
\hyphenation{FontForge MetaPost PostScript ConTeXt bold-italic of-fice prvot-ní prvot-ní-mi splněn plněn plně-ní splně-ní plsti-cí plstí-cí Fran-ti-šek pře-spal}
\mainlanguage[sk]
\hyphenation{FontForge MetaPost PostScript ConTeXt bold-italic of-fice}
\mainlanguage[en]
\hyphenation{FontForge MetaPost PostScript ConTeXt bold-italic of-fice}
\mainlanguage[uk]
\hyphenation{FontForge MetaPost PostScript ConTeXt bold-italic of-fice}

%----------- Languages : quotations ---------------------------------%

\setupdelimitedtext[quotation:1][method=font]
\setupdelimitedtext[quotation:1][left={\symbol[leftquotation]},right={\symbol[rightquotation]}]
\setupdelimitedtext[quotation:2][left={\symbol[leftquote]},right={\symbol[rightquote]}]

%----------- Setup Thesis -------------------------------------------%

\def\setupthesis[#1][#2]{\ctxlua{documentdata.THESIS.setupthesis('#1','#2')}\getparameters[THESIS]['#2']}
\def\setupthesisdesign[#1][#2]{\ctxlua{documentdata.THESIS.setupthesisdesign('#1', '#2')}}
\def\installuni[#1][#2]{\ctxlua{documentdata.THESIS.installuni('#1', '#2')}}

\newif\ifTHESISdesignautocaptionsc\THESISdesignautocaptionscfalse
\newif\ifTHESISdesigncaseautocaptionsc\THESISdesigncaseautocaptionscfalse


\startluacode
ctx=context
documentdata=documentdata or {}
documentdata.THESIS=documentdata.THESIS or {}
T=documentdata.THESIS

T.frontmatter=T.frontmatter or {}
T.design=T.design or {}

function T.err(errmsg)
  inspect("CTX-THESIS ERROR: "..errmsg)
  ctx("\\errmessage{CTX-THESIS ERROR: "..errmsg.."}") 
end

err=T.err

function T.setupthesis(keywords, keyvals)
  keyword_options = utilities.parsers.settings_to_array(keywords)
  named_values = utilities.parsers.settings_to_hash(keyvals)

  T.setup=named_values
  S=T.setup
  S.subtitlesep=" : "
  local a=(keyword_options[1]):split('+')
  S.mainlanguage=a[1]
  S.auxlanguage=a[2]
  local l=S.mainlanguage
  if l=="cz" then S.mainlanguage="cs" end
  if l=="en" then l="uk" end
  ctx("\\mainlanguage["..l.."]")  
  if l=="uk" or l=="us" then S.mainlanguage="en" end

  S.university=keyword_options[2]
  S.faculty=keyword_options[3]
  S.department=keyword_options[4]
  if not T.unilist[S.university] then T.unilist = require "ctx-thesis-listofuni" end
  
  if #keyword_options<2 then err("Missing setup?") end
  if not S.location then err("Missing location.") else
    local l=S.location 
          l=l:split(',') if #l==1 then l[2]=l[1] end 
          S.location=l  
  end

  S.declaration=S.declaration or {}
  S.declaration.cs="\\thesisDeclarationCStext"
  S.declaration.sk="\\thesisDeclarationSKtext"
  S.declaration.en="\\thesisDeclarationENtext"

  S.title=string.split(S.title,S.subtitlesep)
  S.titleen=string.split(S.titleen,S.subtitlesep)


  if not S.auxlanguage then S.auxlanguage="en" end
  if not S.mainlanguage or not T.termlist[S.mainlanguage] then S.mainlanguage="en" end
  if S.mainlanguage=="en" then S.auxlanguage=S.auxlanguage or "cs" end
  ctx("\\def\\refname{"..T.termlist[S.mainlanguage].ref.."}") 
  T.preparedate()
  T.preparesupervisor()
  T.setupthesisdesign("init")
end

function T.setupthesisdesign(keywords, keyvals)
  if keywords ~= "init" then
    keyword_options = utilities.parsers.settings_to_array(keywords)
    named_values = utilities.parsers.settings_to_hash(keyvals)
    for _,x in pairs(named_values) do T.design[_]=x or 1 end

    if T.design.autocaptionsc then ctx.THESISdesignautocaptionsctrue() end

--inspect(keyword_options)
--inspect(named_values)
end
  

  T.design.assignmentalternative=T.design.assignmentalternative or "text"
  T.design.assignmentposition=T.design.assignmentposition or ""
  T.design.assignmentscale=T.design.assignmentscale or ""

  T.design.autoreffont = function (s) if T.design.autocaptionsc then
    local listco={"Převzato.z:","Prevzaté.z:","Zdroj:","Zdroje:"}
    local scxxx=" \\scr "
    for i=1,#listco do
      local co=listco[i]
      local n,m=s:find(co)
      if n then s=s:sub(1,n+co:len())..scxxx..s:sub(n+co:len()+1) end
    end
  end
  return s end

  documentdata.THESIS.design=T.design

  if not T.design.alternative then T.design.alternative="singlesided" end
  T.design.page="" 
  if T.design.alternative=="doublesided" then 
--inspect ( T.design.page) os.exit()
    T.design.page="odd"
    ctx("\\setuppagenumbering[alternative=doublesided,state=start]") 
    ctx("\\setuplayout[location=doublesided]") 
  end
end

function T.installuni(keywords, keyvals)
  kw = utilities.parsers.settings_to_array(keywords) 
  named_values = utilities.parsers.settings_to_hash(keyvals)
  local u=T.unilist
  local modify = kw[1]=="modify"  if modify then table.remove(kw,1) end
  if u[kw[1]] then err("The key "..kw[1].." already exists.") else u[kw[1]] = named_values
  end
  T.unilist=u
end

T.unilist={
	mendelu={en="Mendel University in Brno",cs="Mendelova univerzita v~Brně",sk="Mendelova univerzita v~Brně"},
--
 	af={en="Faculty of Agronomy",cs="Agronomická fakulta",sk="Agronomická fakulta"},
 	frrms={en="Faculty of Regional Development and International Studies",cs="Fakulta regionálního rozvoje a~mezinárodních studií",sk="Fakulta regionálního rozvoje a~mezinárodních studií"},
 	icv={en="Institute of Lifelong Learning",cs="Institut celoživotního vzdělávání",sk="Institut celoživotního vzdělávání"},
	ldf={en="Faculty of Forestry and Wood Technology",cs="Lesnická a dřevařská fakulta",sk="Lesnická a dřevařská fakulta"},
 	pef={en="Faculty of Business and Economics",cs="Provozně ekonomická fakulta",sk="Provozně ekonomická fakulta"},
	zf={en="Faculty of Horticulture",cs="Zahradnická fakulta",sk="Zahradnická fakulta"},
--
 	ui={en="Department of Informatics",cs="Ústav informatiky",sk="Ústav informatiky"},
 	undb={en="Department of Furniture, Furniture Design and Habitation",cs="Ústav nábytku, designu a bydlení",sk="Ústav nábytku, designu a bydlení"},
--
 	none={en="",cs="",sk=""},
}

T.termlist={
	cs={bp="Bakalářská práce",dp="Diplomová práce",pp="Doktorská práce",sp="Seminární práce",zp="Závěrečná práce",
	    pr="Protokol",pt="Projekt",
	    kbp="Komentář k bakalářské práci", kdp="Komentář k diplomové práci", kpp="Komentář k doktorské práci",
	    sv="Vedoucí práce",kw="Klíčová slova",ab="Abstrakt",
	    cp="Čestné prohlášení",ref="Literatura",
	    kde="V",dne="dne",sgn="podpis",
	    ozp="NA MÍSTĚ TOHOTO LISTU\\par SE NACHÁZÍ ORIGINÁL\\par ZADÁNÍ PRÁCE.\\par",
	},
	sk={bp="Bakalárska práca",dp="Diplomová práca",pp="Doktorská práca",sp="Seminárna práca",zp="Zaverečná práca",
        pr="Protokol",pt="Projekt",
	    kbp="Komentár k bakalárskej práci", kdp="Komentár k diplomovej práci", kpp="Komentár k doktorskej práci",
        sv="Vedúci práce",kw="Kľúčové slová",ab="Abstrakt",
	    cp="Čestné prehlásenie",ref="Literatúra",
	    kde="V",dne="dňa",sgn="podpis",
	    ozp="NA MIESTE TOHTO LISTU\\par SA NACHÁDZA ORIGINÁL\\par ZADANIA PRÁCE.\\par",
	},
        en={bp="Bachelor thesis",dp="Diploma thesis",pp="PhD~thesis",sp="Seminar thesis",zp="Final thesis",
	    pr="Report",pt="Project",
	    kbp="Commentary on Bachelor thesis", kdp="Commentary on Diploma thesis", kpp="Commentary on PhD~thesis",
	    sv="Supervisor",kw="Key words",ab="Abstract",
	    cp="Declaration",ref="References",
	    kde="",dne="",sgn="signature",
	    ozp="INSTEAD OF THIS SHEET\\par THERE IS THE ORIGINAL OF THE\\par BACHELOR THESIS ASSIGNMENT.",
	},
}

function T.getuniversity(uf,lang)
  if not lang or lang=="" then lang=T.setup.mainlanguage end
  local abbr=nil
  uf=uf or "u"
  if uf=="u" then abbr=T.setup.university else 
  if uf=="f" then abbr=T.setup.faculty else
  if uf=="d" then abbr=T.setup.department or "none" 
  else err("Undefined university level code.")
  end end end
  if not abbr then err("Missing university/faculty name (abbreviation).") end
  if not T.unilist[abbr] then err("University/faculty abbreviation not defined.") T.unilist[abbr]={} end
  local lang=lang
  if not T.unilist[abbr][lang] then if lang=="sk" then lang="cs" else if lang=="cs" then lang="sk" end end end
  if not T.unilist[abbr][lang] then err("University/faculty name not defined for language "..lang..".") T.unilist[abbr][lang]="???" end
  T.errorgetuni=T.errorgetuni or T.unilist[abbr][lang]:sub(1,1)=="?" 
  return T.unilist[abbr][lang] 
end

function T.getterm(abbr,lang)
  if not abbr then err("Sorry, I am not a wizard.") end
  if not lang then lang=T.setup.mainlanguage end
  return T.termlist[lang][abbr]
end

function T.preparedate()
  local s,y=T.setup,""
  if s.date then 
	s.year="\\date["..s.date.."][year]" 
	s.date="\\date["..s.date.."]" 
    else
    if s.year then s.date="\\currentdate[day,{.~},month,{~}]"..s.year  else 
       s.date="\\currentdate" s.year="\\currentdate[year]"
  end end
  T.setup=s
end
\stopluacode

%%%%%%%%%%%%%%%%%%%%%%%%%%%%%%%%%%%%%


%---------------------------------

%----------- Rozměry stránky a zrcadla ------------------------------%

\def\setupThesisUserlayout{}

\def\setupThesislayout{\setuplayout[%
		width=32cc,
		height=718dd, % 43*14.4+96 (96=8*12 za footer, header atd.)
%	textheight=640dd, %  43*14.4  (nefunguje) 
%		lines=43,  % pův 36
		grid=yes,
%%%%%% NEEEE!!!!!! top=0dd, top=10cc,
		veroffset=-1cc,    % -2cc pro TL2017, pro TL2015 stačilo 0cc
		horoffset=2.75cc,
%		leftmargindistance=0cc,
%		leftmargin=0cc,
%		rightmargin=3cc,
		margin=0cc,
%		leftedge=0cc,
%		leftedgedistance=0cc,
		header=1.5cc,
		footer=1.5cc,
		headerdistance=3cc,
		footerdistance=2cc,
%		location={top,middle,singlesided},
		marking=off,
	]\setupThesisUserlayout
}


\ifx\thesisDeclarationWidth\undefined\def\thesisDeclarationWidth{\makeupwidth}\fi

\def\setupTitlelayout{\setuplayout[veroffset=-1cc,horoffset=0.25cc,width=36cc,top=0cc,height=56cc,footer=0em,header=0dd,footerdistance=0cm,headerdistance=0cm]\setupThesisUserlayout}
\def\setupEmptylayout{\setuplayout[veroffset=-5.4375cc,horoffset=-5.4375cc,width=46.5cc,top=0cc,height=70.5cc,lines=55,footer=0em,header=0dd,footerdistance=0cm,headerdistance=0cm]\setupThesisUserlayout}

\setuppapersize[A4][A4]

%----------- Obecné definice ----------------------------------------%

\setuppositioning[unit=cc]

%\setupframed    [background=color]
\setupframedtext[background=color]
\setupbackground[background=color]

\setupcolors[rgb=no,cmyk=yes,spot=yes,state=start,overprint=yes] 

%----------- Fonty --------------------------------------------------%

\unexpanded
\def\fontthesisUniversity{\tfa}
\def\fontthesisTitle{\definebodyfontenvironment[17dd][interlinespace=30dd]\switchtobodyfont[2cc]\bf}
\def\fontthesisSubtitle{\definebodyfontenvironment[17dd][interlinespace=30dd]\switchtobodyfont[2cc]\tf}
\def\fontthesisAuthor{\tfc}
\def\fontthesisSupervisor{\tfa}
\def\fontthesisLocation{\tfa}
\def\fontthesisTypeName{\bfb}
\def\fontthesisAbstractText{\bfa}
\def\fontthesisAbstractTitle{\it}
\def\fontthesisAbstractAuthor{\sc}

\starttypescript [libertinusx]
  \definetypeface [libertinusx] [rm] [serif] [libertinus][default]
  \definetypeface [libertinusx] [ss] [sans]  [libertinus][default]
  \definetypeface [libertinusx] [tt] [mono]  [libertinus][none][rscale=0.88]
  \definetypeface [libertinusx] [mm] [math]  [libertinus][default]
  \quittypescriptscanning
\stoptypescript

\starttypescript [libertinuslmtt]
  \definetypeface [libertinuslmtt] [rm] [serif] [libertinus][default]
  \definetypeface [libertinuslmtt] [ss] [sans]  [libertinus][default]
  \definetypeface [libertinuslmtt] [tt] [mono]  [modern][none]%[default] 
  \definetypeface [libertinuslmtt] [mm] [math]  [libertinus][default]
  \quittypescriptscanning
\stoptypescript

%----------- Visící interpunkce -------------------------------------%

\setupalign[hz,hanging]	%normal,%nothz,nohz,
\setupparagraphs[align={hz,hanging}]
\setupframedtext[align={hz,hanging}]
% musí být až tady, jinak se visící interpunkce neuplatní
\setupbodyfont[libertinusx]

%----------- Matematika----------------------------------------------%

\setupmathematics[autopunctuation=yes]

\let\ORIGunit=\unit
\def\unit#1{\null\unskip\ORIGunit{#1}}

%----------- Odstavce -----------------------------------------------%

\setupindenting[yes,first,1.25em,always]
\setupfloat[indentnext=yes]

%----------- Poznámky pod čarou -------------------------------------%

\setupnotation[footnote][alternative=serried,distance=0.25em]

%----------- Paginace -----------------------------------------------%

%\def\MyPgnum#1{\raise1cc\hbox{#1}}
\setuppagenumbering     [alternative=singlesided,
                        location={header,right},
%                        command=\MyPgnum,
                        way=bytext, partnumber=no]

%----------- Dělení slov, parchanti a vdovy -------------------------%

\startsetups[grid][thesispenalties]
    \setdefaultpenalties
    \setpenalties\widowpenalties{2}{10000}
    \setpenalties\clubpenalties {2}{10000}
\stopsetups

\setuplayout[grid=yes, setups=thesispenalties]

\brokenpenalty=10000 % nastavuje až se v dokumentu
\def\zlom{\unskip\penalty-10000}

%----------- Obrázky, tabulky ---------------------------------------%

\setuptables[bodyfont=8.5dd,distance=medium,textwidth=\textwidth]

%----------- Tabulky / NEW ---------------------------------%

\setupTABLE[background=color]%,offset=3dd]

\def\newtab{\dotripleempty\doMyNewTab}

\def\doMyNewTab[#type][#loc][#label]#caption#setup#data#legend{\start#setup%
% \if #type   %TODO
  \placetable[#loc][#label]{#caption}{\startalignment[middle]#data\stopalignment\par\startframedtext[frame=off]#legend\stopframedtext}
\stop}

%----------- Tabulky / OBSOLETE ----------------------------%

\setupcaption[table][align={middle,nothyphenated}]
\setupfloat[table][location={middle}]


\def\TABULKA[#1][#2]#3#4\par{% #1=location #2=label #3=caption #4=table
%  \placetable[#1][#2]{#3}{\bTABLE#4\eTABLE\par} vs. 
  \placetable[#1][#2]{#3}{\startalignment[middle]\bTABLE#4\eTABLE\stopalignment\par}
}

\def\TABDATA[#1][#2]#3#4\par{% #1=location #2=label #3=caption #4=table
  \placetable[#1][#2]{#3}{#4\par}
}

\def\TABEMPTY[#1][#2]#3#4{% #1=location #2=label #3=caption #4=table
  \placetable[#1][#2]{#3}{\startcolor[white]*#4*\stopcolor}  % TODO
}

\def\tabulka#1#2#3#4{% #1=label #2=caption #3=format #4=table
  \placetable[][#1]{#2}{\starttable[#3]\HL[3]#4\HL[3]\stoptable}
}

\def\Tabulka#1#2#3#4#5{% #1=label #2=caption #3=format #4=table #5=legend
\placetable[][#1]{#2}{{#3\noindentation\startalignment[middle]\bTABLE#4\eTABLE\stopalignment}\par#5}
%\placetable[#1][#2]{#3}{\bTABLE#4\eTABLE}
%\placetable[][#1]{#2}{\startcombination[1*2]{#3\bTABLE#4\eTABLE}{}{#5}{}\stopcombination\blank}
}

\startluacode
function documentdata.tabCSV(data)
  local sep,n,m = data:sub(1,1),data:find("%d+",2) 
  local ny = tonumber(data:sub(n,m))
  data = string.strip(data:sub(m+1))
  local t = data:split(sep) if sep~=" " then table.remove(t) end
  if #t % ny ~= 0 then ctx("\\errmessage{Problém s počtem sloupců v tabCSV...}") end

  ctx.bTABLE()
  for x=1,#t,ny do ctx.bTR() for y=x,x+ny-1 do ctx.bTD() ctx(t[y]) ctx.eTD() end ctx.eTR() end
  ctx.eTABLE()
end
\stopluacode

\def\tabCSV{\dodoubleempty\doMyTabCSV}

\def\doMyTabCSV[#loc][#label]#caption#setup#data#legend{%
\startplacetable[location=#loc,reference=#label,title=#caption]#setup\par\indenting[no]\null\hfill
  \ctxlua{if buffers.exists("#data") then ctx.getbuffer({"#data"}) else documentdata.tabCSV("#data") end}\hfill\null%
  \startalignment[middle]\par#legend\stopalignment
\stopplacetable
}


%----------- Obrázky / NEW --------------------------------------------%

\setupframed[offset=0cc]

\def\frameson{\setupfloat[frame=on,framecolor=red]%\setupcaption[align=middle,frame=off,framecolor=blue]
	\setupframed[frame=on,framecolor=green]\setupexternalfigure[frame=on,framecolor=black]}
\def\framesoff{\setupfloat[frame=off]\setupcaption[frame=off]\setupframed[frame=off]\setupexternalfigure[frame=off]}
\framesoff

\definefloat[figurecomb][figurecombs][figure]%[location={middle,none}]
%\setupfloat[figurecomb][toffset=2cc,topmargin=2cc] %,leftmargin=10cc,leftmargindistance=10cc]
%\setupcombination[figurecomb][distance=10cc] 
\setupcombination[distance=10cc] 

\setuplabeltext[cs][figurecomb=Obrázek~,figurecomblt=Obrázek~,figureright=Obrázek~]
\setuplabeltext[sk][figurecomb=Obrázok~,figurecomblt=Obrázok~,figureright=Obrázok~]
\setuplabeltext[en][figurecomb=Figure~,figurecomblt=Figure~,figureright=Figure~]

\definefloat[figurecomblt][figurecomblts][figurecomb]%[location={middle,none}]
\definefloat[figurecombrt][][figure]%[location={middle,none}]
\definefloat[figurecomblb][][figure]%[location={middle,none}]
\definefloat[figurecombrb][][figure]%[location={middle,none}]
\definefloat[figurecombmb][][figure]%[location={middle,none}]
\definefloat[figurecombmt][][figure]%[location={middle,none}]
\definefloat[figurepage][figurepages][figure]%[location={middle,none}]

\definefloat[figureright][figurerights][figure]
%\definefloat[figureright][figureure][margin=60pt,rightmargindistance=30pt]

%\setupfloat[figureright][offset=0ex,indentnext=yes,toffset=-1cc,boffset=-1em,spacebefore=0em,spaceafter=4*line]  %,before={\godown[2em]}] % do v0.57: toffset=0
%\setupfloat[figureright][loffset=0ex,roffset=0em,location={right,hanging}]

\setupfloat[figurecomblt][boffset=-2ex,bottommargin=2ex,topmargin=2ex]
\setupfloat[figurecombrt][boffset=-2ex,bottommargin=2ex]

%\def\mycmdfigureright#1{\startalignment[middle]#1\stopalignment}
%\setupcaption[figureure][offset=0cc,align={middle,nothyphenated}] %?
%\setupcaption[figureure,figurecomb][align={middle,nothyphenated},textcommand={\crlf},distance=0cc,spaceafter=medium]%,bottomoffset=2cm]


\setupcaption[figure][align={middle,nothyphenated},textcommand=\crlf,headseparator={\kern-1cc}]

%\setupcaption[figureright][location={bottom},%align={middle,nothyphenated},
%textcommand=\crlf,
%before={\godown[0em]},
%\setupcaption[figureright][
%spaceafter=big,
%spaceinbettw¨
%command={\crlf},
%distance=1em
%]
%]

%\setupcaption[figurepage][textcommand=\crlf]

\def\figureBuf{\dodoubleempty\doMyFigureBuf}

\def\doMyFigureBuf[#loc][#label]#caption#setup#data#legend{%
\startplacefigure[location=#loc,reference=#label,title=#caption]#setup\par\indenting[no]\null\hfill
  \ctxlua{if buffers.exists("#data") then ctx.getbuffer({"#data"}) else ctx("něco jiného") end}\hfill\null%
  \startalignment[middle]\par#legend\stopalignment
\stopplacefigure
}

%----------- Obrázky ----------------------------------------------------%

\def\obrazek{\doifnextcharelse{[}{\doMyFigure}{\doMyFigure[top]}}
\def\obrazekB{\doifnextcharelse{[}{\doMyFigure}{\doMyFigure[bottom]}}
\def\obrazekH{\doifnextcharelse{[}{\doMyFigure}{\doMyFigure[here]}}

\def\obrazekX{\doifnextcharelse{[}{\doMyFigureGeneral}{\doMyFigureGeneral[top]}}
\def\obrazekW{\doifnextcharelse{[}{\doMyFigureW}{\doMyFigureW[top]}}
\let\obrazekw=\obrazekW
%\def\obrazky{\doifnextcharelse{[}{\doMyFigureX}{\doMyFigureX[top]}}
%\def\obrazky{\doMyFigureX}
\def\obrazkyvedlesebe{\dodoubleempty\doMyFigures}
\def\obrazkydvavedlesebe{\doMyFigures[2*1][]}
\def\obrazkydvavedlesebeH{\doMyFigures[2*1][here]}
\def\obrazkydvavedlesebeB{\doMyFigures[2*1][bottom]}

%\setupcaption[figure][align={middle,nothyphenated},textcommand=\crlf,headseparator={\kern-1cc}]

\def\doMyFigure[#loc]#label#caption#fname#extfigparams{%
  \doMyFigureGeneral[#loc]{#label}{#caption}{\externalfigure[#fname][#extfigparams]}
}

\def\obrazky[#combination][#loc]#label#captions#fnames#extfigparams{%
  \doMyFigureGeneral[#loc]{#label}{#captions}{\start
    \setupcombinations[align=middle,distance=1cc,after={\blank[1cc]}]
    %\setupcaption[spaceafter=10cc]
   \startcombination[#combination]
   \ctxlua{
     local np,p=4,{'#label','#captions','#fnames','#extfigparams'}
     for i=3,np do p[i]=p[i]:split(';') end
     for i=1,#combination do
       for j=3,np do p[j][i]=p[j][i]:strip() end
       ctx("{\\externalfigure["..p[3][i].."]["..p[4][i].."]}{}") end
  }\stopcombination\stop
}}


\def\THESISautoreffont#1{\startluacode ctx(T.design.autoreffont('#1')) \stopluacode}

\def\doMyFigureGeneral[#loc]#label#caption#mycode{%
  \startplacefigure[location={#loc},reference={#label},
   title={\THESISautoreffont{#caption}},
   ]{#mycode}\stopplacefigure}

\def\doMyFigureW[#loc]#label#caption#fname#extfigparams{%
  \doMyFigure[#loc]{#label}{#caption}{#fname}{width=\makeupwidth,#extfigparams}}

\def\doMyFigures[#combination][#loc]#labels#captions#fnames#extfigparams{%
  \startluacode 
    function getloc() x='#loc' or "" x=x.strip('#loc') if x=="" then x="top" end return x..",none" end
  \stopluacode
%\setupfloatcaption[figurecomblt,figurecombrt][width=8cm]
  \startplacefigure[location={\ctxlua{ctx(getloc())}}]
%  \setupfloat[boffset=0ex,toffset=2ex]%-0.875cc]%,leftmargin=0cc,rightmargindistance=20cc] %width=fit,loffset=fit,roffset=fit, leftmargindistance=10cc
%  \setupexternalfigure[loffset=0cc,roffset=0cc] % ???
  \setupcombination[width=fit,distance=1cc,align=middle]
  \startfloatcombination[#combination]
    \setupfloatcaption[align=middle]
    \doifstringinstringelse{2021.}{\contextversion}{\setupfloatcaption[spacebefore=1cc]}{}
    \setupcaption[figurecomblt][bottomoffset=1cc]%,distance=100cc, %,topoffset=-2cc,bottomoffset=0cc
    \startluacode
    local figtype="lt" 
    local np,p=4,{'#labels','#captions','#fnames','#extfigparams'}
    for i=1,np do p[i]=p[i]:split(';') end
    for i=1,#combination do
      for j=1,np do p[j][i]=p[j][i]:strip() end
      local pt=utilities.parsers.settings_to_hash(p[4][i])
      if pt.captionwidth then ctx("\\setupfloatcaption[figurecomblt][width="..pt.captionwidth.."]") end
      ctx("\\startplacefigurecomb"..figtype.."[location=middle, reference="..p[1][i]..", title={"..T.design.autoreffont(p[2][i]).."}]")
      ctx("\\externalfigure["..p[3][i].."]["..p[4][i].."]")
      ctx("\\stopplacefigurecomb"..figtype) 
    end
    \stopluacode
%title={"..T.design.autoreffont(p[2][i]).."}]")
%--comb"..figtype)
%      ctx("\\placefigure[]["..p[1][i].."]{"..T.design.autoreffont(p[2][i]).."}{\\externalfigure["..p[3][i].."]["..p[4][i].."]}")
%      ctx("\\placefigure[]["..p[1][i].."]{"..p[2][i].."}{\\externalfigure["..p[3][i].."]["..p[4][i].."]}")
  \stopfloatcombination
  \stopplacefigure}

%----------- Obrázky / TO BE CHECKED ------------------------------------%

\def\obrazekP#1#2#3#4{% #1=label #2=caption #3=název souboru #4=parametry
\startalignment[middle]
  \startplacefigurepage[location=page,reference=#1,title=#2]
	\externalfigure[#3][#4]
  \stopplacefigurepage
\stopalignment}
\let\obrazekp=\obrazekP

\setupexternalfigures[location=default]
\setupcaption[figureright][spaceinbetween=none,width=max,align=middle,textcommand=,spaceafter=big,spacebefore=none]

\def\@obrazekF[#1]#2#3#4#5{% #1=label #2=caption #3=název souboru #4=parametry
  \start\godown[-0.875cc]\blank
  \startplacefigureright[location={#1,-0*line},title={#3},reference=#2,]
  \godown[-\lineheight]
  \externalfigure[#4][#5]
  \stopplacefigureright
  \stop}

\def\obrazekR#label#caption#fname#extfigparams{\@obrazekF[right]{#label}{#caption}{#fname}{#extfigparams}}
\def\obrazekL#label#caption#fname#extfigparams{\@obrazekF[left]{#label}{#caption}{#fname}{#extfigparams}}

\def\obrazekDS#1#2#3#4(#5){% #1=label #2=caption #3=název souboru #4=parametry
  \placefigure[here,force][#1]{#2}{\startpositioning%
	\position(#5){\externalfigure[#3][#4]}%
  \stoppositioning\blank}
}

%----------- Obrázky : English macro names ------------------------------%

\let\figure=\obrazek
\let\figureB=\obrazekB
\let\figureH=\obrazekH
\let\figureW=\obrazekW
\let\figureX=\obrazekX
\let\figureP=\obrazekP
\let\figureL=\obrazekL
\let\figureR=\obrazekR

\let\figuresSBS=\obrazkyvedlesebe
\let\figuresSBStwo=\obrazkydvavedlesebe
\let\figuresSBStwoH=\obrazkydvavedlesebeH
\let\figuresSBStwoB=\obrazkydvavedlesebeB

\let\figures=\obrazky

%----------- Grafy --------------------------------------------------%

\setuplabeltext[cs][chart=Graf~]
\setuplabeltext[sk][chart=Graf~]
\setuplabeltext[en][chart=Chart~]


%----------- Kódy ---------------------------------------------------%

\definecolor[bggray][c=0,m=0,y=0,k=0.07]
\definecolor[frgray][c=0,m=0,y=0,k=0.15]

\setupstartstop[DefaultSnippet][before=,after=,style=]
\defineframedtext[MPtyping][background=color,backgroundcolor=bggray,
   			width=\makeupwidth,
			offset=2dd,loffset=6dd,roffset=6dd,
   			frame=on,
			framecolor=frgray,
   			]

%\definetextbackground[MPtextbg][background=color,backgroundcolor=bggray,frameoffset=0cc,frame=on,
%	location=paragraph,leftoffset=0cc,topoffset=0cc,bottomoffset=0cc,]
\setuptyping[MP][before=\startMPtyping,after=\stopMPtyping,			
%\setuptyping[MP][before=\startMPtextbg,after=\stopMPtextbg,			
	indentnext=auto,
]


\definefloat[code][codes][figure]
\definecombinedlist[codes][code][level=chapter]

\setuplabeltext[cs][code=Zdrojový~kód~]
\setuplabeltext[sk][code=Zdrojový~kód~]
\setuplabeltext[en][code=Source code~]

\def\typesourcecode{\dodoubleempty\doTypeSourceCode}
\def\sourcecode{\dodoubleempty\doSourceCode}

\def\doTypeSourceCode[#loc][#label]#caption#setup#buffer#legend{\start#setup%
\startplacecode[location=#loc,reference=#label,title=#caption]
  \typebuffer[#buffer]
  \startalignment[middle]\par#legend\stopalignment
\stopplacecode
\stop}

\def\doSourceCode[#loc][#label]#caption#setup#buffer#legend{\start#setup%
\startplacecode[location=#loc,reference=#label,title=#caption]
  \getbuffer[#buffer]
  \startalignment[middle]\par#legend\stopalignment
\stopplacecode
\stop}

%----------- Seznamy ------------------------------------------------%

%\definesymbol[5][--]
\definesymbol[10][=]

%----
\setupitemize
  [each]
  [placestopper=no,
       lefttext=,
      righttext=,
           left=,
          right=,
]

\startluacode

local rightsymbol = {
    en = { a = ")", n = ")", I = ")", i = ")", A = ")" },
    cs = { a = ")", n = ".", I = ".", i = ".", A = "."},
    sk = { a = ")", n = ".", I = ".", i = ".", A = "."},
}
local leftsymbol = {
    en = { a = "(", n = "(", I = "(", A = "(", i = "(" },
}

leftsymbol["en-gb"] = leftsymbol.en
leftsymbol["en-us"] = leftsymbol.en
leftsymbol["uk"] = leftsymbol.en
rightsymbol["en-gb"] = rightsymbol.en
rightsymbol["en-us"] = rightsymbol.en
rightsymbol["uk"] = leftsymbol.en

local righttextsymbol = rightsymbol
local lefttextsymbol  = leftsymbol

local function itemizeleft(language,symbol)
    local left = leftsymbol[language] and leftsymbol[language][symbol]
    context(left or "")
end
local function itemizelefttext(language,symbol)
    local left = lefttextsymbol[language] and lefttextsymbol[language][symbol]
    context(left or "")
end

local function itemizeright(language,symbol)
    local right = rightsymbol[language] and rightsymbol[language][symbol]
    context(right or ".")
end
local function itemizerighttext(language,symbol)
    local right = righttextsymbol[language] and righttextsymbol[language][symbol]
    context(right or ".")
end

interfaces.implement {
    name      = "ItemizeRight",
    public    = true,
    actions   = itemizeright,                                                                                                                                   
    arguments = "2 strings",
}

interfaces.implement {
    name      = "ItemizeLeft",
    public    = true,
    actions   = itemizeleft,
    arguments = "2 strings",
}

interfaces.implement {
    name      = "ItemizeRightText",
    public    = true,
    actions   = itemizerighttext,
    arguments = "2 strings",
}

interfaces.implement {
    name      = "ItemizeLeftText",
    public    = true,
    actions   = itemizelefttext,
    arguments = "2 strings",
}

\stopluacode

\setupitemize
  [each]
  [packed,joindup,autointro]
  [symalign=left, 
   distance=0.5em,
   stopper=\ItemizeStopper{\currentlanguage}{\currentitemgroupsymbol},
   left=\ItemizeLeft{\currentlanguage}{\currentitemgroupsymbol},
   lefttext=\ItemizeLeft{\currentlanguage}{\currentitemgroupsymbol},
   right=\ItemizeRight{\currentlanguage}{\currentitemgroupsymbol},
   righttext=\ItemizeRight{\currentlanguage}{\currentitemgroupsymbol},
   leftmargin=\parindent,
   width=1.5em,]  

%---

\definedescription[desc][
   headstyle=\bf, style=normal, 
   alternative={hanging}, 
   width=fit,
   indenting=yes,distance=0.2em,stretch=0.5,shrink=0.9,
   margin=\parindent,
   after={\godown[-\lineheight]},
%.1667em
]

%----------- Vyznačení autorů v textu -------------------------------%

\let\internalauthoremph=\sc%mallcaps
\def\scc#1,{{\internalauthoremph #1},}
\def\scl#1({{\internalauthoremph #1}(}
\def\scr#1){{\internalauthoremph #1})}
\def\scx#1|{{\internalauthoremph #1}} 
\def\podpis#1{\rightaligned{\em#1}}


%----------- Titulky ------------------------------------------------%

\definehead[partpage][chapter]
\definehead[kap][chapter]
\definehead[singlekap][chapter]
\definehead[kapx][title]
\definehead[pkap][section]
\definehead[ppkap][subsection]
\definehead[pppkap][subsubsection]
\definehead[ppppkap][subsubsubsection]
\definehead[pkapx][subject]
\definehead[ppkapx][subsubject]
\definehead[pppkapx][subsubsubject]
\definehead[ppppkapx][subsubsubsubject]
\definehead[BIBSTYrefs][kap]% nebo chapter?

%\setupheads[align={right,nothyphentated,nohz},indentnext=yes]
%\setuphead[kap][align={right,nothyphentated,nohz},indentnext=yes]
\setuphead[kap,singlekap,pkap,ppkap,pppkap,ppppkap,BIBSTYrefs][align={right,nothyphenated,stretch},indentnext=yes]
\setuphead[kapx,pkapx,ppkapx,pppkapx,ppppkapx][align={right,nothyphenated,stretch},indentnext=yes]

%\setupheader[emptyheaderwithoutline][frame=off,header=empty]
%\setupheader[headerwithline][frame=off,bottomframe=on]

\setuphead[partpage][style=\bfc,header=nomarking,incrementnumber=list] %(own)number=no %,command={\writetolist[kapx]{#2}}]
%beforesection={\godown[12cc]}]

\setuphead[kap,chapter,BIBSTYrefs,kapx][style=\bfc,sectionnumber=yes,header=nomarking,]
%\setuphead[kap,chapter][numbercommand=\gobbleoneargument]
\setuphead[pkap,section,pkapx,subject][style=\bfb,before={\blank[8*small]},after={\blank[4*small]}]%,beforesection={\blank[0*small]},aftersection={\blank[0*small]},]
\setuphead[ppkap,subsection,ppkapx,subsubject][style=\tfa,before={\blank[4*small]},after={\blank[4*small]}]%,beforesection={\blank[0*small]},aftersection={\blank[0*small]},]
\setuphead[pppkap,subsubsection,pppkapx,subsubsubject][style={\normal},before={\blank[4*small]},after={\blank[4*small]},]%beforesection={\blank[0*small]},aftersection={\blank[0*small]}]
\setuphead[ppppkap,subsubsubsection,ppppkapx,subsubsubsubjet][style=\it,after={\blank[0*small]}]
%\setuphead[pppkapx,subsubsubject][before={\blank[2*medium]},after={\blank[1*medium]},style=\it,]
\setuphead[BIBSTYrefs][number=no]


\def\thesisheadertextsoff{\setupheadertexts[text][][]}
\def\thesisheadertextson{\setupheadertexts[text][{\getmarking[chapter]}][pagenumber]}

\def\cast#1{\page{%
	\thesisheaderlinesoff
   	\thesisheadertextsoff
%\setupheadertexts[text][][]
	%\startframedtext[loffset=6cc,toffset=12cc,location=middle]
	\startpartpage[title={\uppercase{#1}}]\stoppartpage
	%\stopframedtext
	\page}
	\thesisheaderlineson
   	\thesisheadertextson
%\setupheadertexts[text][rrrr][pagenumber]%{\getmarking[chapter]}][pagenumber]
}


%----------- Obsah --------------------------------------------------%

\definecombinedlist[obsah][partpage,chapter,singlekap,kap,kapx,pkap,ppkap,pppkap,ppppkap][level=chapter]

\setuplist[singlekap][headnumber=yes,
	before={\blank[0*medium]\godown[0dd]},
	after={\page[no]\blank[0*medium]\godown[0dd]\noindenting\thinrule}
	]

\setupheadtext[cs][obsah=Obsah]
\setupheadtext[sk][obsah=Obsah]
\setupheadtext[en][obsah=Contents]

\setuplist[partpage][headnumber=no,pagenumber=no,margin=0cc,before={\blank[2*big]%\godown[2dd]
  },after={\godown[-9dd]\noindenting\thinrule\godown[10dd]\pagebreak[no]},
	style=bold,alternative=b]

%\def\mylistkap#1#2#3{#1/#2/#3}
%\def\mylistkapt#1{#1//}

\setuplist[BIBSTYrefs,chapter,kap,kapx,title][headnumber=yes,
	before={\blank[big]}, %\godown[1dd]
    after={\godown[-9dd]		\noindenting\thinrule\godown[10dd]},
%    textcommand=\mylistkapt,
%    command=\mylistkap,
	style=bold,alternative=b,
]

\setuplist[BIBSTYrefs]		[headnumber=no]
\setuplist[pkap,section]
  [headnumber=yes,style=normal,alternative=c,margin=0.75em,distance=1.375em,before={\godown[1dd]},after={\godown[-0dd]},
  ]
\setuplist[ppkap,subsection]
  [headnumber=yes,style=normal,alternative=c,margin=2em,distance=1.75em,before={\godown[1dd]},after={\godown[0dd]}
  ]
\setuplist[pppkap,subsubsection][headnumber=yes,style=normal,alternative=c,margin=3.25em,distance=-1.5em,before={\godown[1dd]},after={\godown[0dd]}]
\setuplist[ppppkap,subsubsubsection][headnumber=yes,style=slanted,alternative=c,margin=4.5em,distance=-1.875em,before={\godown[1dd]},after={\godown[0dd]}]

\setuplist[pkap][width=1.33cc]
\setuplist[ppkap][width=2cc]

%----------- Seznamy, zkratky ---------------------------------------%

%\definesynonyms[abbreviation][abbreviations][\infull]
\setupsynonyms[abbreviation][textstyle=smallcaps]
\setupheadtext[cs][figures=Seznam obrázků]
\setupheadtext[sk][figures=Zoznam obrázkov]
\setupheadtext[en][figures=List of Figures]
\setupheadtext[cs][charts=Seznam grafů]
\setupheadtext[sk][charts=Zoznam grafov]
\setupheadtext[en][charts=List of Charts]
\setupheadtext[en][tables=List of Tables]
\setupheadtext[cs][tables=Seznam tabulek]
\setupheadtext[sk][tables=Zoznam tabuliek]
\setupheadtext[cs][abbreviations=Seznam použitých zkratek]
\setupheadtext[sk][abbreviations=Zoznam použitých skratiek]
\setupheadtext[en][abbreviations=List of Abbreviations]
\setupheadtext[en][codes=List of Source Codes]
\setupheadtext[cs][codes=Seznam zdrojových kódů]
\setupheadtext[sk][codes=Zoznam zdrojových kódov]


%\setupcombinedlist[content][list={chapter,section,subsection}]
\setuplist[table,figure,figurecomb,figurecomblt,figurepage,figureright][state=start,alternative=b,headnumber=yes,criterium=all]
\setuplist[charts][state=start,list={chart,figurecomb,figurecomblt,figurecombrt,figurepage,figureright}]

\definecombinedlist[tables] [table]	[level=chapter]
\definecombinedlist[codes]  [code]  	[level=chapter]
\definecombinedlist[figures][figure,figurecomb,figurecomblt,figurecombrt,figurepage,figureright][level=chapter]
\definecombinedlist[charts] [chart,figurecomblt,figurecombrt,figurepage,figureright] [level=chapter]

\setupcaption[table][location=top]
\setupcaptions[align={nothyphentated}]

\def\THESIScompletelistof#listname{%
  \kap{\headtext{#listname}}
  \doifinsetelse{#listname}{figures,charts,codes,tables}
	{\csname place#listname\endcsname[criterium=all]}
    {
	  \doifinsetelse{#listname}{abbreviations}
	  {\csname placelistof#listname\endcsname}
	  {\csname placelistof#listname\endcsname}
	}
  \writebetweenlist[kap]{}%\godown[-14.4dd]}
}
\setupsectionblock[backpart][before={\writebetweenlist[kap]{\blank[big]}}]

%--------- Pohlaví ---------------------------------------------------%

\def\Female#1#2{\if M\thesisAuthorGender#1\else #2\fi}

%--------- Prohlášení ------------------------------------------------%

\def\thesisDeclarationCStext{Prohlašuji, že jsem práci {\it\thesisTitle}
vypracoval\Female{}{a} samostatně a~veškeré použité prameny a~informace uvádím v~seznamu
použité literatury. Souhlasím, aby moje práce byla zveřejněna v~souladu s~§\,47b
zákona č.\,111/1998~Sb., o~vysokých školách, ve znění pozdějších předpisů a~v~souladu s~platnou
Směrnicí o~zveřejňování vysokoškolských závěrečných prací. 
Prohlašuji, že tištěná podoba závěrečné práce a elektronická podoba
závěrečné práce zveřejněná v~aplikaci Závěrečné práce v~Univerzitním
informačním systému je identická. 
\par 
Jsem si vědom\Female{}{a}, že se na moji práci vztahuje zákon č.\,121/2000~Sb., autorský zákon, 
a~že \ctxlua{ctx(T.getuniversity("u"))} má právo na uzavření
licenční smlouvy a~užití této práce jako školního díla podle §\,60 odst.\,1 autorského zákona.
\par 
Dále se zavazuji, že před sepsáním licenční smlouvy o~využití díla jinou osobou
(subjektem) si vyžádám písemné stanovisko univerzity, že předmětná licenční smlouva není
v~rozporu s~oprávněnými zájmy univerzity, a~zavazuji se uhradit případný příspěvek na
úhradu nákladů spojených se vznikem díla, a~to až do jejich skutečné výše.}

\def\thesisDeclarationSKtext{Prehlasujem, že som prácu {\it\thesisTitle}
vypracoval\Female{}{a} samostatne a~všetky použité zdroje a~informácie uvádzam v~zozname 
použitej literatúry. Súhlasím, aby moja práca bola zverejnená v~súlade §\,47b zákona č.\,111/1998~Sb., 
o~vysokých školách, v~znení neskorších predpisov a~v~súlade s~platnou Směrnicí
o~zveřejňování vysokoškolských závěrečných prací. Prehlasujem, že tlačená
podoba záverečnej práce a elektronická podoba záverečnej práce zverejnená 
v~aplikácii Závěrečné práce
v~Univerzitním informačním systému je identická.
\par
Som si vedom\Female{ý}{á}, že sa na moju prácu vzťahuje zákon č.\,121/2000~Sb., autorský zákon, 
a~že \ctxlua{ctx(T.getuniversity("u"))} má právo na uzatvorenie licenčnej zmluvy
a~použitie tejto práce ako školského diela podľa §\,60 odst.\,1 autorského zákona.
\par
Ďalej sa zaväzujem, že pred spísaním licenčnej zmluvy o~použití diela inou osobou 
(subjektom) si vyžiadam písomné stanovisko univerzity, že predmetná licenčná zmluva
nie je v rozpore s~oprávnenými záujmami univerzity a zaväzujem sa uhradiť prípadný
príspevok na úhradu nákladov spojených so vznikom diela, a~to až do ich skutočnej výšky.}

\def\thesisDeclarationENtext{I hereby declare that this thesis entitled {\it\thesisTitleEN} 
was written and completed by me. I also declare that all the sources and
information used to complete the thesis are included in the list of references. 
I agree that the thesis could be made public in accordance with Article~47b of Act
No.~111/1998~Coll., Higher Education Institutions and on Amendments and
Supplements to Some Other Acts (the Higher Education Act), and in accordance with the
current Directive on publishing of the final thesis.
I declare
that the printed version of the thesis and electronic version of the thesis
published in the application %of the 
Final 
Thesis in the University Information System is identical.
\par
I am aware that my thesis is written in accordance to Act No.~121/2000~Coll.
(the Copyright Act) and therefore
\ctxlua{ctx(T.getuniversity("u"))} has the right to conclude licence
agreements on the utilization of the thesis as a school work in accordance
with Article~60~(1) of the Copyright Act.
\par
Before concluding a licence agreement on utilization of the work by another
person, I will request a written statement from the university that the licence agreement
is not in contradiction to legitimate interests of the university, and I will also pay a
prospective fee to cover the cost incurred in creating the work
to the full amount of such costs.}

%--------- Typ práce a další texty------------------------------------%

\def\thesisSupervisorText{\ctxlua{ctx(T.getterm("sv"))}}
\def\thesisAbstractText{\ctxlua{ctx(T.getterm("ab"))}}
\def\thesisKeywordsText{\ctxlua{ctx(T.getterm("kw"))}}
\def\thesisDeclarationText{\ctxlua{ctx(T.getterm("cp"))}}

\def\thesisKeywordsEN{\ctxlua{ctx(T.setup.keywordsen)}}
\def\thesisKeywords{\ctxlua{ctx(T.setup.keywords)}}
\def\thesisAbstractEN{\ctxlua{ctx(T.setup.abstracten)}}
\def\thesisAbstract{\ctxlua{ctx(T.setup.abstract)}}
\def\thesisDeclaration{\ctxlua{ctx(T.setup.declaration[T.setup.mainlanguage])}}

\def\thesisLocation{\ctxlua{ctx(T.setup.location[1])}}
%\def\thesisYear{\currentdate[year]}
%\def\thesisDate{\currentdate}
\def\thesisYear{\ctxlua{ctx(T.setup.year)}}
\def\thesisDate{\ctxlua{ctx(T.setup.date)}}


\def\thesis{\currentdate}

\def\thesisTypeName{\ctxlua{ctx(T.getterm(T.setup.type))}}

\def\thesisUniversity{\ctxlua{ctx(T.getuniversity("u"))}}
\def\thesisFaculty{\ctxlua{ctx(T.getuniversity("f"))}}
\def\thesisDepartment{\ctxlua{ctx(T.getuniversity("d"))}}

\def\thesisTitle{\ctxlua{ctx(T.setup.title[1])    
   if T.setup.title[2] then ctx(T.setup.subtitlesep..T.setup.title[2]) end}}
\def\thesisTitleEN{\ctxlua{ctx(T.setup.titleen[1])
   if T.setup.titleen[2] then ctx(T.setup.subtitlesep..T.setup.titleen[2]) end}}

\def\thesisAcknowledgement{\ctxlua{ctx(T.setup.acknowledgement)}}

% logika: bp+sv=>def  bp=>err  xx+sv=>def  xx=>nic
\startluacode
function T.preparesupervisor()
local types={bp=1,dp=1,pp=1}
local s=T.setup
local sv=T.setup.supervisor
if sv and sv~="" then ctx("\\def\\thesisSupervisor{\\framed[frame=off]{"..sv.."}}") 
else if types[s.type] then ctx("\\def\\thesisSupervisor{xxxxxx}") T.err("Missing supervisor's name...") end
end
end
\stopluacode
%\def\thesisSupervisor{\ctxlua{ctx(T.setup.supervisor)}}

\def\thesisAuthorSurname{\ctxlua{ctx(T.setup.authorsurname)}}
\def\thesisAuthorName{\ctxlua{ctx(T.setup.authorname)}}
\def\thesisAuthorDegree{\ctxlua{ctx(T.setup.authordegree)}}
\def\thesisAuthorGender{\ctxlua{ctx(T.setup.authorgender)}}

\startluacode
local function join_strings (s1,s2,sep)
  sep = sep or ""
  local joined = s1
  if s2 then joined = joined..sep..s2 end
  return joined
end

function T.getthesistitle (sep)
  sep = sep or T.setup.subtitlesep 
  return join_strings(T.setup.title[1],T.setup.title[2],sep)
end

function T.getthesistitleen (sep)
  sep = sep or T.setup.subtitlesep 
  return join_strings(T.setup.titleen[1],T.setup.titleen[2],sep)
end

function T.getthesistitlemeta (sep, sepu)  -- todo
  sepu = sepu or " / "
  return join_strings( T.getthesistitle (sep), join_strings(T.setup.titleen[1],T.setup.titleen[2],sep), sepu)
end

function T.getthesisauthor (sep, reverse)
  if reverse then mysep = ", " else mysep = " " end
  sep = sep or mysep
  if reverse 
    then return join_strings(T.setup.authorsurname,T.setup.authorname,sep)
    else return join_strings(T.setup.authorname,T.setup.authorsurname,sep)
  end
end

function T.getthesisauthormeta (sep, sepu)
  sep = sep or " " 
  sepa = sepa or " : "
  sepu = sepu or " / "
  return join_strings(T.getthesisauthor(sep),join_strings(T.getuniversity("u",T.setup.mainlanguage),T.getuniversity("u",T.setup.auxlanguage),sepu),sepa)
end

function T.getthesistype (main)
   return T.getterm(T.setup.type,main or T.setup.mainlanguage)
end

function T.getthesistypemeta (sep)
  sep = sep or " / "
  return join_strings(T.getthesistype(),T.getthesistype(T.setup.auxlanguage),sep) or ""
end

function T.getthesiskeywords (sep)
  sep = sep or " ; "
  return join_strings(T.setup.keywords,T.setup.keywordsen,sep) 
end
\stopluacode

\def\thesisoddpage{\ctxlua{ctx.page({T.design.page})}}

\def\thesisTitlePage{{\setuplayout[grid=no]\noindentation%
%\null\par\godown[0cc]
%if T.mainlanguage=="en" 
%then 
%ctx(T.getthesistitle("\\par\\fontthesisSubtitle "))
%else 
%ctx(T.getthesistitleen("\\par\\fontthesisSubtitle ")) 
%end
  \def\thesisTitle{\ctxlua{
if T.setup.mainlanguage=="en" then
ctx(T.getthesistitleen("\\par\\fontthesisSubtitle ")) 
else
ctx(T.getthesistitle("\\par\\fontthesisSubtitle ")) 
end
}}
  \def\titnl{\crlf}{\fontthesisUniversity\thesisUniversity\par\thesisFaculty\par\if*\else\thesisDepartment\par\fi\par}
  \godown[-4dd]{\black\thinrule}\blank[3*big]\vfill%
  {\fontthesisTitle\thesisTitle\par}\par\blank[10*medium]
  {\fontthesisTypeName\thesisTypeName}\vfill~\vfill%\blank[3*big]  
  {\fontthesisAuthor\thesisAuthorDegree~\thesisAuthorName~\thesisAuthorSurname}\crlf\blank[14*big]%
	\ifx\thesisSupervisor\undefined\else{\fontthesisSupervisor\thesisSupervisorText:\crlf\thesisSupervisor}\fi\crlf\blank[3*medium]
  {\fontthesisLocation\thesisLocation~\thesisYear}
}}

\def\thesisAcknowledgementPage{\thesisoddpage\null\vfill\null\quad\startframedtext\thesisAcknowledgement\stopframedtext}
\def\thesisDeclarationPage{\thesisoddpage
%~\vfill
~\par\blank[16*big]
\startframedtext[location=none,align=tolerant,frame=off,depthcorrection=off,width=\thesisDeclarationWidth]
{\fontthesisAbstractText\thesisDeclarationText}\par\blank\setuplayout[grid=yes]\thesisDeclaration
\par%crlf
\blank[2*big]%\noindenting
\startluacode
local s=T.setup
local ml=s.mainlanguage
local g=T.getterm
if ml=="en" then kde=1 else kde=2 end
ctx(g("kde",ml).."~"..s.location[kde].." "..g("dne",ml).." "..s.date)
ctx("\\hfill{\\tfxx\\periods[30]}\\crlf\\null\\hfill")
ctx(g("sgn",ml))
\stopluacode\hskip3em\null\stopframedtext
%\godown[-.33cc]
}


\def\thesisAssignmentForm{\startluacode 
local t=T.setup.thesisassignmentform
local s=T.termlist[T.setup.mainlanguage]
local d=T.design
inspect(d) 

ctx.page({d.page}) 
ctx.setupThesislayout()

if t and t~="" then 
  local ts=t:split(',')
  local width="width=\\textwidth" 
  local scale=d.assignmentscale:mysplit(',')
  local pos=d.assignmentposition:mysplit(',')

  for i=1,table.getn(pos),2 do pos[i+1]=pos[i]..","..pos[i+1] end

  for i=1,table.getn(ts) do 
    ctx.noindenting()
    if d.assignmentalternative=="page" then ctx("\\setupEmptylayout") end
    if not scale[i] 
        then scale[i]=1000 
        else if d.assignmentalternative~="text" then width="" end
    end
    if not pos[i*2] then pos[i*2]="(0,0)" end
    ctx.startpositioning()
    ctx("\\position"..pos[i*2].."{\\externalfigure["..ts[i]:strip().."][frame=on,scale="..scale[i]..","..width.."]}")
    ctx.stoppositioning()
    ctx.page() 
  end
    ctx("\\setupThesislayout")
else
  ctx("\\gdef\\framedtextOZP{\\framedtextozp{"..s.ozp.."}}")
  ctx("\\framedtextOZP{}")
end
\stopluacode}

\defineframedtext[framedtextozp][location=middle,align=middle,width=\makeupwidth,frame=on,offset=1cc]

\def\thesisAbstracts{\page
%\noindenting
  \def\pomAbstract##1##2##3##4##5##6##7{
  \startframedtext{\noindenting\fontthesisAbstractText##1}\crlf\blank{%
       \fontthesisAbstractAuthor\noindenting\thesisAuthorSurname,~\thesisAuthorName}. 
      {\fontthesisAbstractTitle##3}. \ctxlua{if T.setup.type~="xp" then ctx("##2.") end}
       \thesisLocation%
%\startluacode
%--inspect(T.setup.type, T.setup.university, '##7', T.getuniversity("u",##7))
%inspect(not(T.setup.type=="xp" and T.setup.university=="none"))
%\stopluacode
%--os.exit()
       \ctxlua{if not(T.setup.type=="xp" and T.setup.university=="none") then ctx(" : "..T.getuniversity("u",##7)) end},
       \thesisYear.\par\blank[2*big]##6\par\blank[2*big]{\noindenting\fontthesisAbstractText##4}
       \par\blank[big]\noindenting##5\stopframedtext}
%
  {\startluacode pomlang=T.setup.mainlanguage  T.setup.mainlanguage="en"\stopluacode
  \pomAbstract{\thesisAbstractText}{\thesisTypeName}{\thesisTitleEN}{\thesisKeywordsText}{\thesisKeywordsEN}{\thesisAbstractEN}{"en"}
  \startluacode T.setup.mainlanguage=pomlang \stopluacode
  }
  {\startluacode 
	if T.setup.mainlanguage=="en" 
	  then ctx("\\language["..T.setup.auxlanguage.."]") 
		T.setup.mainlanguage=T.setup.auxlanguage 
	end
   \stopluacode
%-- 	else ctx(T.setup.mainlanguage) 
  \par\blank[3*big]\null\quad%
  \pomAbstract{\thesisAbstractText}{\thesisTypeName}{\thesisTitle}{\thesisKeywordsText}{\thesisKeywords}{\thesisAbstract}{""}
%  \ctxlua{T.setup.mainlanguage="en"}
  }
}

\setupframedtext[width=\makeupwidth,location=center,frame=none,offset=0cc,margin=0cc,indenting={yes,first,1.25em}]

%%%%%%%%%%%%%%%%%%%%%%%

\startluacode
T.displayfrontmatter = function ()
  ctx.startfrontmatter()
  local f=T.frontmatter
  for p=1,#f do if p>1 then ctx.page() end ctx(f[p]) end
  ctx.stopfrontmatter()
end

T.myswap = function (t,i,j) i=tonumber(i) j=tonumber(j)
  t[i],t[j]=t[j],t[i]
  return(t)
end

table.removebylist=function(t,list)
  local a=string.split(list,',') table.sort(a)
  for i=table.getn(a),1,-1 do table.remove(t,tonumber(a[i])) end
  return(t)
end
\stopluacode

\def\insertfrontmatterpage{\dosingleempty\doinsertfrontmatterpage}
\def\removefrontmatterpages[#1]{\ctxlua{T.frontmatter=table.removebylist(T.frontmatter,'#1')}}
\let\removefrontmatterpage=\removefrontmatterpages
\def\swapfrontmatterpages[#1,#2]{\ctxlua{T.frontmatter=T.myswap(T.frontmatter,'#1','#2')}}
\let\swapfrontmatterpage=\swapfrontmatterpages
\def\doinsertfrontmatterpage[#1]#2{\iffirstargument\ctxlua{table.insert(T.frontmatter,'#1','#2')}\else\ctxlua{table.insert(T.frontmatter,'#2')}\fi}
\def\displayfrontmatter{\ctxlua{T.displayfrontmatter()}}

\def\displaythesisTitlePage{{\setupTitlelayout\startalignment[middle]{\thesisTitlePage}\stopalignment}\page}

\def\displaythesisAbstracts{
   {\setupframedtext[width=\makeupwidth,location=middle,frame=none] %margin=6cc
    \thesisAbstracts
}}

\def\thesisContents{\setupThesislayout
  \setuppagenumbering[state=start]
  \thesisheaderlineson
  \thesisheadertextson
   {\def\titnl{\space}\def\tocnl{\crlf}
    \setupwhitespace[-1dd]
    \completeobsah
   }}


\def\thesisheaderlinesoff{\setupbackgrounds[header][text][frame=off,bottomframe=off]}
\def\thesisheaderlineson{\setupbackgrounds[header][text][frame=off,bottomframe=on]}

\def\thesisBase{%
  \ctxlua{if T.errorgetuni then ctx("\\errmessage{Problem with university/faculty name.}") end}
\setupbodyfont[12dd] 
  \thesisheaderlinesoff 
  \setuppagenumbering[state=stop]
  \setupindenting[yes,first,1.25em]
  \brokenpenalty=10000
  \startfrontmatter
    \thesisheaderlinesoff 
    \displayfrontmatter
    \setuppagenumbering[state=start]
    \thesisheaderlineson
  \stopfrontmatter
  \setupThesislayout
  \setuppagenumbering[state=start]
  \thesisheaderlineson
%  \setupheadertexs[{\namedheadnumber{chapter}.\kern0.5em\getmarking[chapter]}][\pagenumber]
  \thesisheadertextson
%  \setupheadertexts[{\getmarking[chapter]}][\pagenumber]
}

%--------- Thesis ----------------------------------------------------%

\definestartstop[thesis][before={\thesisPDFAnecessities\thesisBase
},commands={},after={}]
\insertfrontmatterpage{\\showlayout\\displaythesisTitlePage\\setuplayout[grid=yes]}
\insertfrontmatterpage{\\thesisAssignmentForm}
\insertfrontmatterpage{\\thesisAcknowledgementPage}
\insertfrontmatterpage{\\thesisDeclarationPage}
\insertfrontmatterpage{\\displaythesisAbstracts}
\insertfrontmatterpage{\\thesisContents}

%---------PDF/A ------------------------------------------------------%

\def\thesisPDFAnecessities{
  \setupbackend[
    format={pdf/a-2a:2005}, % or pdf/a-2b:2005
    profile={default_cmyk.icc,default_rgb.icc,default_gray.icc},
%    export=yes,
 ]
  \setupstructure[state=start,method=auto]
  \def\comma{‚}
  \enabledirectives[interaction.identity.preroll]
  \startluacode function comma2comma(r) return r:gsub(',',"\\comma\\ ") end 
  \stopluacode
   \setupinteraction
    [title={\ctxlua{ctx(tex2txt(T.getthesistitlemeta()))}},
     subtitle={\ctxlua{ctx(tex2txt(T.getthesistypemeta()))}},
     author={\ctxlua{ctx(tex2txt(T.getthesisauthormeta()))}},
     keyword={\ctxlua{ctx(comma2comma(tex2txt(T.getthesiskeywords())))}},
  ]
}

%--------- Interaction------------------------------------------------%

\setupinteraction [color=green, style=bold]

%--------- The end ---------------------------------------------------%

\let\startbackmatterorig=\startbackmatter
\def\startbackmatter{\startbackmatterorig\def\titnl{\space}\def\tocnl{\crlf}}

%--------- Temporary patches------------------------------------------%

% since 0.79 till TL2021 (?): % odstraněno 2021
%\unprotect
%\setfalse\c_page_floats_pack_flushed
%\protect
% since 0.80 till TL2021 (?):
\unexpanded\def\resetbreakpoints
  {\exhyphenchar\hyphenasciicode % 2020.03.05
   \attribute\breakpointattribute\attributeunsetvalue}
% since 0.81 till TL2021 (?):
% comma in PDF/A
% since 0.91 till ??:
% \doiftringinstring{2021.}
\endinput

%====================================================================%
% Thesis in ConTeXt                                                  %     
%--------------------------------------------------------------------%
% Sazební styl pro ConTeXt -- (c) TH, 2013--2022                     %
%====================================================================%
