%
\usemodule[ctx-thesis-v0.971]
\usemodule[bib.sty-v2.73]
\usemodule[vim]

\def\courierfont{\setupbodyfont[courier,12.1dd]}  % font definice
\definecolor[codesnippetbackground][x=ffffff]
\definevimtyping[SWIFT][syntax=swift, before={\startframedtext[width=\makeupwidth, background=color, backgroundcolor=codesnippetbackground, frame=off, style={\switchtobodyfont[small]\setupinterlinespace}]}, after={\stopframedtext}, numbering=yes]
\definevimtyping[YAMLNUM][syntax=yaml, before={\startframedtext[width=\makeupwidth, background=color, backgroundcolor=codesnippetbackground, frame=off, style={\switchtobodyfont[small]\setupinterlinespace}]}, after={\stopframedtext}, numbering=yes]
\definevimtyping[YAML][syntax=yaml, before={\startframedtext[width=\makeupwidth, background=color, backgroundcolor=codesnippetbackground, frame=off, style={\switchtobodyfont[small]\setupinterlinespace}]}, after={\stopframedtext}, numbering=no]

\setupthesis[cs,mendelu,pef,none][ % jazyk,univerzita,fakulta,ústav/katedra/pracoviště ; language,university,faculty,department
  type={bp},                 % bp,dp,pp,zp,sp,pr,pt aj./etc.
  authorname={Jan},	     % jméno
  authorsurname={Janovec},        % příjmení
  authordegree={},	     % titul před jménem
  authorgender={M},          % pohlaví (holky mají F)
  supervisor={Ing. David Procházka, Ph.D.},        % vedoucí práce
  title={Mobilní aplikace pro pomoc imigrantům},                           % název práce
  titleen={Mobile application to help the immigrants}, 	           % název práce anglicky
  keywords={},    %
  keywordsen={}, %
  acknowledgement={},	           % poděkování
  abstract={},		                           % český abstrakt
  abstracten={},		                   % anglický abstrakt
  location={Brno,Brně},	   % místo vydání (za čárkou 6. pád) ; location (second parameter is not necessary for English)
%  year={2021},		   % rok odevzdání práce (automaticky aktuální rok) ; year, the default is the current year
% thesisassignmentform={img/zadani.png},  % seznam souborů se skenem zadání práce; file is thesis assignment
]
\startthesis
\startbodymatter
\kap{Ještě jsem měl problém tady}
{\courierfont UserDefaults} a UserDefaults
\par
Nejedná se o nic vážného, jenom mi přijde zvláštní, že když nastavím velikost na 12dd, tak mi font nefunguje. Jinak se používá v celém dokumentu velikost 12, že?
\bbib

\publW{
	\autor{Alvarez, Pedro.}
	\nazev{Medium}
	\nazevdok{iOS Architectures Explained: Which One Best Fits My Project?}
	\www{https://betterprogramming.pub/ios-architectures-explained-which-one-best-fits-my-project-94b4ffaad16}
	\online{2022-01-28}
	\rok{2021}
}

\publW{
	\autor{Wikipedia contributors.}
	\nazev{Wikipedia}
	\nazevdok{Swift (programming language)}
	\www{https://en.wikipedia.org/wiki/Swift_(programming_language)}
	\online{2022-01-12}
	\rok{2023}
}

\publW{
	\autor{Apple.}
	\nazev{Apple Developer Documentation}
	\nazevdok{UIKit}
	\www{https://developer.apple.com/documentation/uikit}
	\online{2023-01-20}
	\rok{2023}
}

\publW{
	\autor{Jeroen, L.}
	\nazev{Stream}
	\nazevdok{UIKit vs. SwiftUI: How to Choose the Right Framework for Your App}
	\www{https://getstream.io/blog/uikit-vs-swiftui/}
	\online{2023-01-20}
	\rok{2022}
}

\publW{
	\autor{Apple.}
	\nazev{Apple Developer Documentation}
	\nazevdok{SwiftUI}
	\www{https://developer.apple.com/documentation/swiftui}
	\online{2023-01-20}
	\rok{2023}
}

\publW{
	\autor{Hudson, Paul.}
	\nazev{Hacking with Swift}
	\nazevdok{Storing user settings with UserDefaults}
	\www{https://www.hackingwithswift.com/books/ios-swiftui/storing-user-settings-with-userdefaults}
	\online{2023-01-28}
	\rok{2022}
}

\publW{
	\autor{iOS App Templates Contributors.}
	\nazev{iOS App Templates}
	\nazevdok{iOS Data Persistence in Swift}
	\www{https://iosapptemplates.com/blog/ios-development/data-persistence-ios-swift}
	\online{2023-01-28}
	\rok{2021}
}

\publW{
	\autor{Apple.}
	\nazev{Apple Developer Documentation}
	\nazevdok{Core Data}
	\www{https://developer.apple.com/documentation/coredata}
	\online{2023-01-28}
	\rok{2023}
}

\publW{
	\autor{Apple.}
	\nazev{Apple Developer Documentation}
	\nazevdok{MapKit}
	\www{https://developer.apple.com/documentation/mapkit/}
	\online{2023-02-03}
	\rok{2023}
}

\publW{
	\autor{Rudrank, Riyam.}
	\nazev{Semaphore}
	\nazevdok{UI Testing in Swift}
	\www{https://semaphoreci.com/blog/ui-testing-swift}
	\online{2023-02-10}
	\rok{2021}
}

\publW{
	\autor{Code with Chris.}
	\nazev{Code with Chris}
	\nazevdok{Xcode UI Testing in Swift – Code Examples}
	\www{https://codewithchris.com/xcode-ui-testing-swift/}
	\online{2023-02-10}
	\rok{2021}
}

\publW{
	\autor{Wikipedia contributors.}
	\nazev{Wikipedia}
	\nazevdok{Chatbot}
	\www{https://en.wikipedia.org/wiki/Chatbot}
	\online{2023-02-20}
	\rok{2023}
}

\publW{
	\autor{MessageKit Contributors.}
	\nazev{GitHub}
	\nazevdok{MessageKit}
	\www{https://github.com/MessageKit/MessageKit}
	\online{2023-02-20}
	\rok{2023}
}

\publW{
	\autor{Hudson, Paul.}
	\nazev{Hacking with Swift}
	\nazevdok{Introducing MVVM into your SwiftUI project}
	\www{https://www.hackingwithswift.com/books/ios-swiftui/introducing-mvvm-into-your-swiftui-project}
	\online{2023-03-16}
	\rok{2022}
}

\publS{
	\autorkorp{Bundesamt fuer Migration und Fluechtlinge.}
	\nazev{Ankommen}
	\verze{1.7.5}
	\rok{2022}
	\www{https://apps.apple.com/cz/app/ankommen/id1066804488}
	\online{}
}

\publS{
	\autorkorp{Heinrich & Reuter Solutions GmbH.}
	\nazev{Welcome app Germany}
	\verze{2.1.0}
	\rok{2020}
	\www{https://apps.apple.com/cz/app/welcome-app-germany/id1047174574}
	\online{}
}


%CITATION📚CITATION📚CITATION📚CITATION📚CITATION📚CITATION📚CITATION📚CITATION📚CITATION📚CITATION📚
\ebib


\stopbodymatter

%%%%%%%%%%%%%%%%%%%%%%%% Varianta, kdy seznamy jsou součástí práce a nejsou uvedeny v přílohách

\setupsectionblock[backmatter][before={\setuplist[kap][before={}]}]

\startbackmatter

\THESIScompletelistof{tables}
\THESIScompletelistof{figures}
%\THESIScompletelistof{abbreviations}
\THESIScompletelistof{codes}

\stopbackmatter
\stopthesis

\endinput		

%%%% TODO %%%%%%%%%%%%%%%%%%%%%%%%%%%%%%
Tady si můžeš psát poznámky, které se neobjeví ve výstupu.
